\documentclass[openany,12pt,oneside]{book} % with 'openany' option, no blank page will be left.

\usepackage[T1]{fontenc}
\usepackage[utf8x]{inputenc} 
%\usepackage[italian]{babel} % useful only if the text is in italian

\usepackage[nottoc]{tocbibind} % listoffigures, listoftables and bibliography in the table of contents (TOC)
\usepackage{amsmath} % math package
\usepackage[table,xcdraw]{xcolor} % in order to use colors in tables
\usepackage{multirow} % in order to join cells in tables
\usepackage{makecell}

% Source: https://tex.stackexchange.com/questions/302594/citation-inside-a-caption-dont-follow-order-of-appearance
% prevents cites in captions from misnumering your references.
\usepackage{notoccite}

\usepackage[numbers]{natbib} % Source: https://it.overleaf.com/learn/latex/Bibliography_management_with_natbib
\bibliographystyle{unsrt} % Source: https://www.bibtex.com/s/bibliography-style-base-unsrt/

% Source https://tex.stackexchange.com/questions/146911/alternatives-to-asterisk-and-star-for-superscripts
% in order to use $\ssymbol{n}$ with n=[0,9]
\makeatletter
\newcommand{\ssymbol}[1]{^{\@fnsymbol{#1}}}
\makeatother


\usepackage[a4paper,top=2.5cm,bottom=2.5cm,left=3cm,right=2cm]{geometry}  % custom margins
\linespread{1.5} % change interline spacing (https://en.wikipedia.org/wiki/Leading) to 1.5

\usepackage[inline, shortlabels]{enumitem} % in order to create inline lists
\usepackage{booktabs} % in order to use \bottomrule
%https://tex.stackexchange.com/questions/88929/vertical-table-lines-are-discontinuous-with-booktabs
\aboverulesep=0ex
\belowrulesep=0ex

\usepackage{longtable}

% Source: https://tex.stackexchange.com/questions/463167/table-align-single-cell-without-using-multicolumn
%in order to use \Centering in tables
\usepackage{ragged2e}


\usepackage{graphicx} % add figures with \begin{figure} [...] \end{figure}
\usepackage{float} % in order to use [H] option on figures.
% This option "draw" the figure near to the text where it is added.
% However, if you want to be more flexible, let's use [!htbp] instead of [H].

\usepackage{pdfpages} % \includepdf to include PDF file
\usepackage{hyperref}
\hypersetup{
	colorlinks=true, % set true if you want colored links
	linktoc=all,     % set to all if you want both sections and subsections linked 
	linkcolor=blue,  % set color link (with \ref command)
	urlcolor=blue, % set URL color (with \href command)
	citecolor=blue % set citation color (with \cite command)
}

\usepackage[all]{hypcap} % when you click on a \ref, you will see directly the element (at the beginning) and not only the caption.
% Otherwise you should move up to see, e.g., a figure.

\usepackage{verbatim} % multi-line comments with \begin{comment} [...] \end{comment}


% commands to handle table of contents (TOC)
%tocdepth -> TOC-depth (e.g. \setcounter{tocdepth}{1} you will see 'chapter' and 'section' in the TOC)
%secnumdepth -> up to where TOC should contain the numbers (e.g. \setcounter{secnumdepth}{4} set numbers until 'paragraph')
\setcounter{tocdepth}{2}
\setcounter{secnumdepth}{2}
% Note: the sectioning levels have the following numbers:
% 	-1 part     
% 	0 chapter
% 	1 section
% 	2 subsection
% 	3 subsubsection
% 	4 paragraph
% 	5 subparagraph

\pagestyle{plain} % number of pages below